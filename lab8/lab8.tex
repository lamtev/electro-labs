\documentclass[a4paper,14pt]{extarticle}

\usepackage[utf8x]{inputenc}
\usepackage[T1,T2A]{fontenc}
\usepackage[russian]{babel}
\usepackage{hyperref}
\usepackage{indentfirst}
\usepackage{here}
\usepackage{array}
\usepackage{graphicx}
\usepackage{caption}
\usepackage{subcaption}
\usepackage{chngcntr}
\usepackage{amsmath}
\usepackage{amssymb}
\usepackage{pgfplots}
\usepackage{pgfplotstable}
\usepackage[left=2cm,right=2cm,top=2cm,bottom=2cm,bindingoffset=0cm]{geometry}

\renewcommand{\le}{\ensuremath{\leqslant}}
\renewcommand{\leq}{\ensuremath{\leqslant}}
\renewcommand{\ge}{\ensuremath{\geqslant}}
\renewcommand{\geq}{\ensuremath{\geqslant}}
\renewcommand{\epsilon}{\ensuremath{\varepsilon}}
\renewcommand{\phi}{\ensuremath{\varphi}}

\counterwithin{figure}{section}
\counterwithin{equation}{section}
\counterwithin{table}{section}
\newcommand{\sign}[1][5cm]{\makebox[#1]{\hrulefill}} % Поля подписи и даты
\graphicspath{{pics/}} % Путь до папки с картинками
\captionsetup{justification=centering,margin=1cm}
\def\arraystretch{1.3}

\begin{document}

\begin{titlepage}
\begin{center}
	\textbf{Санкт-Петербургский Политехнический Университет \\Петра Великого}\\[0.3cm]
	\small Институт компьютерных наук и технологий \\[0.3cm]
	\small Кафедра компьютерных систем и программных технологий\\[4cm]
	
	\textbf{ОТЧЕТ}\\ \textbf{о лабораторной работе}\\[0.5cm]
	\textbf{<<Исследование однокаскадных транзисторных усилителей>>}\\[0.1cm]
	\textbf{Электротехника и Электроника}\\[10.5cm]
\end{center}

\begin{flushright}
	\begin{minipage}{0.60\textwidth}
		\begin{flushleft}
			\small \textbf{Работу выполнили студенты}\\[3mm]
			\small группа 23501/4 \hspace*{17mm} Дьячков В.В.\\[3mm]
			\small группа 23501/4 \hspace*{17mm} Ламтев А.Ю.\\[5mm]
			
			\small \textbf{Преподаватель}\\[5mm]
		 	\small \sign[3.5cm] \hspace*{8mm} к.т.н., доц. Кочетков Ю.Д.\\[0.5cm]
		\end{flushleft}
	\end{minipage}
\end{flushright}

\vfill

\begin{center}
	\small Санкт-Петербург\\
	\small \the\year
\end{center}
\end{titlepage}

\section{Цель работы}

Приобретение навыков настройки и исследования импульсных генераторов (автоколебательных и ждущих), определения областей применения различных интегральных микросхем в генераторах.

\section{Чертеж схемы исследуемого устройства}

\begin{figure}[H]
\begin{center}
	\begin{subfigure}[b]{0.35\textwidth}
		\includegraphics[scale=0.3]{scheme1}
		\caption{Симметричный мультивибратор}
	\end{subfigure}
	~
	\begin{subfigure}[b]{0.35\textwidth}
		\includegraphics[scale=0.3]{scheme2}
		\caption{Несимметричный мультивибратор}
	\end{subfigure}
	~
	\begin{subfigure}[b]{0.35\textwidth}
		\includegraphics[scale=0.3]{scheme3}
		\caption{Ждущий генератор}
	\end{subfigure}
	\caption{}
\end{center}
\end{figure}


\section{Исходные данные}

Операционный усилитель \verb+К140УД6+.

\begin{table}[H]
\begin{center}
	\caption{Исходные данные}
	\def\tabcolsep{10pt}
	\begin{tabular}{|c|c|c|c|c|c|c|c|}
		\hline
		$t_{\text{и1}}$, мкс &
		$t_{\text{и2}}$, мкс &
		$K_{\text{Д}}$ &
		$C$, нФ &
		$R_{\text{Д}}$, кОм &
		$C_{\text{Д}}$, пФ &
		$E_{01}$, В &
		$E_{02}$, В \\
		\hline
		20 &
		40 &
		1 &
		3 &
		10 &
		1500 &
		15 &
		-15 \\
	    \hline	
	\end{tabular}
\end{center}
\end{table}

\section{Теоретические расчёты}

\subsection{Расчёт параметров элементов}

$K_\text{Д} = 1 \Rightarrow$ $R_1 = R_2$. Пусть $R_1 = 20$ кОм, тогда $R_2 = 20$ кОм

\begin{displaymath}
	R_3 = \frac{t_{\text{и1}}}{C \cdot ln(1 + \frac{2R_1}{R_2})} = \frac{20 \cdot 10^{-6}}{3 \cdot 10^9 \cdot ln(1 + 2)} = 6 \text{ кОм}
\end{displaymath}

\begin{displaymath}
	R_4 = \frac{t_{\text{и2}}}{C \cdot ln(1 + \frac{2R_1}{R_2})} = \frac{40 \cdot 10^{-6}}{3 \cdot 10^9 \cdot ln(1 + 2)} = 12 \text{ кОм}
\end{displaymath} 

\subsection{Расчёт $t_\text{и}$ симметричного мультивибратора}

\begin{equation}
	t_{\text{и1}} = R_3 C ln\left(1 + \frac{\left(1 - \frac{U_\text{вых}^{-}}{U_\text{вых}^{+}}\right) R_1}{R_2}\right) 
\end{equation}

\begin{displaymath}
	t_{\text{и1}} = 6 \cdot 10^3 \cdot 3 \cdot 10^{-9} \cdot ln\left(1 + \frac{\left(1 - \frac{-15}{0.7 \cdot 15}\right)\cdot 20 \cdot 10^3}{20 \cdot 10^3}\right) = 22.2 \text{ мкс}
\end{displaymath}

\begin{equation}
	t_{\text{и2}} = R_3 C ln\left(1 + \frac{\left(1 - \frac{U_\text{вых}^{+}}{U_\text{вых}^{-}}\right) R_1}{R_2}\right) 
\end{equation}

\begin{displaymath}
	t_{\text{и2}} = 6 \cdot 10^3 \cdot 3 \cdot 10^{-9} \cdot ln\left(1 + \frac{\left(1 - \frac{0.7 \cdot 15}{-15}\right)\cdot 20 \cdot 10^3}{20 \cdot 10^3}\right) = 17.9 \text{ мкс}
\end{displaymath}

\begin{displaymath}
	t_{\text{и}} = \frac{t_{\text{и1}} + t_{\text{и2}}}{2} = \frac{22.2 + 17.9}{2} = 20.1 \text{ мкс}
\end{displaymath}

\subsection{Расчёт $t_\text{и}$ несимметричного мультивибратора}

\begin{equation}
	t_{\text{и1}} = R_3 C ln\left(1 + 2 \cdot \frac{R_1}{R_2}\right) 
\end{equation}

\begin{displaymath}
	t_{\text{и1}} = 6 \cdot 10^3 \cdot 3 \cdot 10^{-9} \cdot ln\left(1 + 2 \cdot \frac{20 \cdot 10^3}{20 \cdot 10^3} \right) = 19.8 \text{ мкс}
\end{displaymath}

\begin{equation}
	t_{\text{и2}} = R_4 C ln\left(1 + 2 \cdot \frac{R_1}{R_2}\right) 
\end{equation}

\begin{displaymath}
	t_{\text{и2}} = 12 \cdot 10^3 \cdot 3 \cdot 10^{-9} \cdot ln\left(1 + 2 \cdot \frac{20 \cdot 10^3}{20 \cdot 10^3} \right) = 39.6 \text{ мкс}
\end{displaymath}

\subsection{Расчёт $t_\text{и}$ ждущего генератора}

\begin{equation}
	t_{\text{и}} = R_3 C ln\left(1 + \frac{R_1}{R_2}\right) 
\end{equation}

\begin{displaymath}
	t_{\text{и}} = 6 \cdot 10^3 \cdot 3 \cdot 10^{-9} \cdot ln\left(1 + \frac{20 \cdot 10^3}{20 \cdot 10^3}\right) = 12.5 \text{ мкс} 
\end{displaymath}

\begin{equation}
	t_{\text{в}} = R_3 C ln\left(1 + \frac{\frac{R_1}{R_2}}{1 + \frac{R_1}{R_2}}\right)
\end{equation}

\begin{displaymath}
	t_{\text{в}} = 6 \cdot 10^3 \cdot 3 \cdot 10^{-9} \cdot ln\left(1 + \frac{\frac{20 \cdot 10^3}{20 \cdot 10^3}}{1 + \frac{20 \cdot 10^3}{20 \cdot 10^3}}\right) = 7.30 \text{ мкс}
\end{displaymath}

\section{Экспериментально снятые зависимости}

\section{Погрешности}

\section{Выводы}

Приведённые погрешности полученных в ходе эксперимента значений $K_1$, $K_2$ и $K_3$ не превышают превышают предельно допустимые погрешности.

Таким образом, формулы \ref{eq:4:1} -- \ref{eq:4:5} являются верными.

\end{document}