\documentclass[a4paper,14pt]{extarticle}

\usepackage[utf8x]{inputenc}
\usepackage[T1,T2A]{fontenc}
\usepackage[russian]{babel}
\usepackage{hyperref}
\usepackage{indentfirst}
\usepackage{here}
\usepackage{array}
\usepackage{graphicx}
\usepackage{caption}
\usepackage{subcaption}
\usepackage{chngcntr}
\usepackage{amsmath}
\usepackage{amssymb}
\usepackage{pgfplots}
\usepackage{pgfplotstable}
\usepackage[left=2cm,right=2cm,top=2cm,bottom=2cm,bindingoffset=0cm]{geometry}

\renewcommand{\le}{\ensuremath{\leqslant}}
\renewcommand{\leq}{\ensuremath{\leqslant}}
\renewcommand{\ge}{\ensuremath{\geqslant}}
\renewcommand{\geq}{\ensuremath{\geqslant}}
\renewcommand{\epsilon}{\ensuremath{\varepsilon}}
\renewcommand{\phi}{\ensuremath{\varphi}}

\counterwithin{figure}{section}
\counterwithin{equation}{section}
\counterwithin{table}{section}
\newcommand{\sign}[1][5cm]{\makebox[#1]{\hrulefill}} % Поля подписи и даты
\graphicspath{{pics/}} % Путь до папки с картинками
\captionsetup{justification=centering,margin=1cm}
\def\arraystretch{1.3}

\begin{document}

\begin{titlepage}
\begin{center}
	\textbf{Санкт-Петербургский Политехнический Университет \\Петра Великого}\\[0.3cm]
	\small Институт компьютерных наук и технологий \\[0.3cm]
	\small Кафедра компьютерных систем и программных технологий\\[4cm]
	
	\textbf{ОТЧЕТ}\\ \textbf{о лабораторной работе}\\[0.5cm]
	\textbf{<<Исследование однокаскадных транзисторных усилителей>>}\\[0.1cm]
	\textbf{Электротехника и Электроника}\\[10.5cm]
\end{center}

\begin{flushright}
	\begin{minipage}{0.60\textwidth}
		\begin{flushleft}
			\small \textbf{Работу выполнили студенты}\\[3mm]
			\small группа 23501/4 \hspace*{17mm} Дьячков В.В.\\[3mm]
			\small группа 23501/4 \hspace*{17mm} Ламтев А.Ю.\\[5mm]
			
			\small \textbf{Преподаватель}\\[5mm]
		 	\small \sign[3.5cm] \hspace*{8mm} к.т.н., доц. Кочетков Ю.Д.\\[0.5cm]
		\end{flushleft}
	\end{minipage}
\end{flushright}

\vfill

\begin{center}
	\small Санкт-Петербург\\
	\small \the\year
\end{center}
\end{titlepage}

\section{Цель работы}

Настройка и расчет масштабирующих устройств, постоянного и переменного токов с двумя и одним источником питания, источника тока, управляемого напряжением, алгебраического сумматора.

\section{Чертеж схемы исследуемого устройства}

\begin{figure}[H]
\begin{center}
	\begin{subfigure}[b]{0.45\textwidth}\centering
		\includegraphics[scale=0.35]{scheme1}
		\caption{Операционный усилитель с инверсией фазы}
	\end{subfigure}
	\begin{subfigure}[b]{0.45\textwidth}\centering
		\includegraphics[scale=0.37]{scheme2}
		\caption{Операционный усилитель без инверсии фазы}
	\end{subfigure}
	\begin{subfigure}[b]{0.6\textwidth}\centering
		\includegraphics[scale=0.48]{scheme3}
		\captionsetup{justification=centering}
		\caption{Схема неинвертирующего усилителя с разделяющим конденсатором}
	\end{subfigure}
	\caption{}
\end{center}
\end{figure}

\newpage

\section{Исходные данные}

Операционный усилитель \verb+К140УД6+.

\begin{table}[H]
\begin{center}
	\caption{Исходные данные}
	\def\tabcolsep{50pt}
	\begin{tabular}{|c|c|c|c|c|c|c|c|c|c|}
		\hline
		$K$ &
		$f$, Гц &
		$I_\text{н}$, мА \\
		\hline
		1 &
		500 &
		0.5 \\
	    \hline	
	\end{tabular}
	\label{tabular:1}
\end{center}
\end{table}

\section{Теоретические расчёты}

Были выбраны следующие элементы:
\begin{center}
	$R_1 = 10$ кОм
	
	$R_2 = K\cdot R_1 = 10$ кОм

	$R_3 = R_1 = 10$ кОм

	$R_4 = R_2 = 10$ кОм
\end{center}

Коэффициенты передачи для исследуемых схем при $K_u \rightarrow \infty$:
\begin{equation}\label{eq:k1}
	K = -\frac{R_2}{R_1} = -\frac{10\text{ кОм}}{10\text{ кОм}} = -1 
\end{equation}
\vspace{-0.75cm}
\begin{equation}\label{eq:k2}
	K = \left(1+\frac{R_2}{R_1}\right) \cdot \frac{R_4}{R_3 + R_4} = \left(1+\frac{10\text{ кОм}}{10\text{ кОм}}\right) \cdot \frac{10\text{ кОм}}{10\text{ кОм} + 10\text{ кОм}} = 1
\end{equation}

Частоту перегибу $f$ можно вычислить по формуле: 
\begin{equation}
f = \frac{1}{2 \pi \cdot C \cdot (R_3 + R_4)}
\end{equation}

Для схемы неинвертирующего усилителя с разделяющим конденсатором была рассчитана емкость конденсатора: \\
\begin{equation}
	C = \frac{1}{2\pi \cdot f\cdot (R_3 + R_4)} = \frac{1}{2\pi \cdot 500\cdot  (10^4 + 10^4)} = 1.59\cdot 10^{-8}\text{ Ф}
\end{equation}

\section{Экспериментально снятые зависимости}

\subsection{Операционный усилитель с инверсией фазы}

\subsubsection{Постоянный входной сигнал}

\begin{table}[H]
\begin{center}
	\caption{Зависимость напряжения $U_\text{вых}$ от $e_c$ инвертирующего усилителя при положительном входном сигнале}
	\def\tabcolsep{40pt}
	\def\arraystretch{1.2}
	\pgfplotstabletypeset[col sep=comma,
	    columns={e_c,u_out,k},
	    column type/.add={|c|}{},
	    columns/e_c/.style={fixed, precision=3, zerofill, column name={$e_c$, В}},
	    columns/u_out/.style={fixed, precision=3, zerofill, column name={$U_\text{вых}$, В}},
	    columns/k/.style={fixed, precision=3, zerofill, column name={$K$}},
	    every nth row={1}{before row=\hline},
	    every head row/.style={before row=\hline, after row=\hline},
	    every last row/.style={after row=\hline}
	   ]{data/1_k_dc_plus.csv}
\end{center}
\end{table}

Среднее значение коэффициента передачи $\bar{K} \approx -0.972$.

\begin{table}[H]
\begin{center}
	\caption{Зависимость напряжения $U_\text{вых}$ от $e_c$ инвертирующего усилителя при отрицательном входном сигнале}
	\def\tabcolsep{40pt}
	\def\arraystretch{1.2}
	\pgfplotstabletypeset[col sep=comma,
	    columns={e_c,u_out,k},
	    column type/.add={|c|}{},
	    columns/e_c/.style={fixed, precision=3, zerofill, column name={$e_c$, В}},
	    columns/u_out/.style={fixed, precision=3, zerofill, column name={$U_\text{вых}$, В}},
	    columns/k/.style={fixed, precision=3, zerofill, column name={$K$}},
	    every nth row={1}{before row=\hline},
	    every head row/.style={before row=\hline, after row=\hline},
	    every last row/.style={after row=\hline}
	   ]{data/1_k_dc_minus.csv}
\end{center}
\end{table}

Среднее значение коэффициента передачи $\bar{K} \approx -1.001$.

\begin{figure}[H]
\begin{center}
	\begin{tikzpicture} [every plot/.append style={thick}]
		\begin{axis}[
			height=0.45\textheight,
			width=0.95\textwidth,
			legend pos=north east,
			xlabel={$e_c$, В},
			ylabel={$U_\text{вых}$, В},
			xlabel near ticks,
			ylabel near ticks,
			axis x line = middle,
			axis y line = left,
			xtick={-16, -12, -8, -4, 0, 4, 8, 12, 16},
			xmin = -16,
			xmax = 16,
			ymin = -15,
			ymax = 15,
			grid=major
		]
		\addplot+[blue, mark options={fill=blue}] table[x=e_c,y=u_out,col sep=comma]{data/1_k_dc_plus.csv};
		\addplot+[blue, mark options={fill=blue}] table[x=e_c,y=u_out,col sep=comma]{data/1_k_dc_minus.csv};
		\addplot +[mark=none, red, samples=2] coordinates {(-15,15) (15,-15)};
		\legend{Эксп., ,Teор.}
		\end{axis}
	\end{tikzpicture}
	\caption{Зависимость напряжения $U_\text{вых}$ от $e_c$ инвертирующего усилителя}
\end{center}
\end{figure}

Среднее значение коэффициента передачи при положительном и отрицательном входном сигнале $\bar{K} \approx -0.956$.

\subsubsection{Перменный входной сигнал}

\begin{table}[H]
\begin{center}
	\caption{Зависимость напряжения $U_\text{вых}$ от $e_c$ инвертирующего усилителя при переменном входном сигнале}
	\def\tabcolsep{40pt}
	\def\arraystretch{1.2}
	\pgfplotstabletypeset[col sep=comma,
	    columns={e_c,u_out,k},
	    column type/.add={|c|}{},
	    columns/e_c/.style={fixed, precision=3, zerofill, column name={$e_c$, В}},
	    columns/u_out/.style={fixed, precision=3, zerofill, column name={$U_\text{вых}$, В}},
	    columns/k/.style={fixed, precision=3, zerofill, column name={$K$}},
	    every nth row={1}{before row=\hline},
	    every head row/.style={before row=\hline, after row=\hline},
	    every last row/.style={after row=\hline}
	   ]{data/1_k_ac.csv}
\end{center}
\end{table}

Среднее значение $\bar{K} \approx 0.996$.

Итого, среднее значение коэффициента передачи при положитеьном, отрицательном и переменном входном сигнале $|\bar{K}| \approx 0.984$

\begin{figure}[H]
\begin{center}
	\begin{tikzpicture} [every plot/.append style={thick}]
		\begin{axis}[
			height=0.45\textheight,
			width=0.95\textwidth,
			legend pos=north west,
			xlabel={$e_c$, В},
			ylabel={$U_\text{вых}$, В},
			xlabel near ticks,
			ylabel near ticks,
			axis x line = middle,
			axis y line = left,
			xmin = 0,
			xmax = 12,
			ymin = 0,
			ymax = 12,
			grid=major
		]
		\addplot table[x=e_c,y=u_out,col sep=comma]{data/1_k_ac.csv};
		\addplot +[mark=none, red, samples=2] coordinates {(1,1) (11.5,11.5)};
		\addplot +[mark=none, dashed, black, samples=2] coordinates {(9.8,0) (9.8,12)};
		\legend{Эксп., Teор.}
		\end{axis}
	\end{tikzpicture}
	\caption{Зависимость напряжения $U_\text{вых}$ от $e_c$ инвертирующего усилителя при переменном входном сигнале}
\end{center}
\end{figure}

\begin{table}[H]
\begin{center}
	\caption{Зависимость коэффициента передачи $K$ от $R_2$ инвертирующего усилителя при переменном входном сигнале}
	\def\tabcolsep{30pt}
	\def\arraystretch{1.2}
	\pgfplotstabletypeset[col sep=comma,
	    columns={r2,e_c,u_out,k},
	    column type/.add={|c|}{},
	    columns/r2/.style={fixed, precision=3, column name={$R_2$, Ом}},
	    columns/e_c/.style={fixed, precision=3, zerofill, column name={$e_c$, В}},
	    columns/u_out/.style={fixed, precision=3, zerofill, column name={$U_\text{вых}$, В}},
	    columns/k/.style={fixed, precision=3, zerofill, column name={$K$}},
	    every nth row={1}{before row=\hline},
	    every head row/.style={before row=\hline, after row=\hline},
	    every last row/.style={after row=\hline}
	   ]{data/1_k_ac_r2.csv}
\end{center}
\end{table}

%\vspace{-1cm}

\begin{figure}[H]
\begin{center}
	\begin{tikzpicture} [every plot/.append style={thick}]
		\begin{axis}[
			height=0.4\textheight,
			width=0.95\textwidth,
			legend pos=north west,
			xlabel={$R_2$, Ом},
			ylabel={$K$},
			xlabel near ticks,
			ylabel near ticks,
			axis x line = middle,
			axis y line = left,
			xtick={0, 200, 400, 600, 800, 1000, 1200, 1400},
			xmin = 0,
			xmax = 1400,
			ymin = 0,
			ymax = 90,
			grid=major
		]
		\addplot+[smooth] table[x=r2,y=k,col sep=comma]{data/1_k_ac_r2.csv};
%		\addplot table[x=e_c,y=u_out,col sep=comma]{data/ampl_theory.csv};
		\legend{Эксп., Teор.}
		\end{axis}
	\end{tikzpicture}
	\caption{Зависимость коэффициента передачи $K$ от $R_2$ инвертирующего усилителя при переменном входном сигнале}
\end{center}
\end{figure}

\subsection{Операционный усилитель без инверсии фазы}

\subsubsection{Постоянный входной сигнал}

\begin{table}[H]
\begin{center}
	\caption{Зависимость напряжения $U_\text{вых}$ от $e_c$ инвертирующего усилителя при положительном входном сигнале}
	\def\tabcolsep{40pt}
	\def\arraystretch{1.2}
	\pgfplotstabletypeset[col sep=comma,
	    columns={e_c,u_out,k},
	    column type/.add={|c|}{},
	    columns/e_c/.style={fixed, precision=3, zerofill, column name={$e_c$, В}},
	    columns/u_out/.style={fixed, precision=3, zerofill, column name={$U_\text{вых}$, В}},
	    columns/k/.style={fixed, precision=3, zerofill, column name={$K$}},
	    every nth row={1}{before row=\hline},
	    every head row/.style={before row=\hline, after row=\hline},
	    every last row/.style={after row=\hline}
	   ]{data/2_k_dc_plus.csv}
\end{center}
\end{table}

%Среднее значение коэффициента передачи $\bar{K} \approx 0.985$.

\begin{table}[H]
\begin{center}
	\caption{Зависимость напряжения $U_\text{вых}$ от $e_c$ инвертирующего усилителя при отрицательном входном сигнале}
	\def\tabcolsep{40pt}
	\def\arraystretch{1.2}
	\pgfplotstabletypeset[col sep=comma,
	    columns={e_c,u_out,k},
	    column type/.add={|c|}{},
	    columns/e_c/.style={fixed, precision=3, zerofill, column name={$e_c$, В}},
	    columns/u_out/.style={fixed, precision=3, zerofill, column name={$U_\text{вых}$, В}},
	    columns/k/.style={fixed, precision=3, zerofill, column name={$K$}},
	    every nth row={1}{before row=\hline},
	    every head row/.style={before row=\hline, after row=\hline},
	    every last row/.style={after row=\hline}
	   ]{data/2_k_dc_minus.csv}
\end{center}
\end{table}

Среднее значение коэффициента передачи $\bar{K} \approx 0.998$.

\begin{figure}[H]
\begin{center}
	\begin{tikzpicture} [every plot/.append style={thick}]
		\begin{axis}[
			height=0.45\textheight,
			width=0.95\textwidth,
			legend pos=north west,
			xlabel={$e_c$, В},
			ylabel={$U_\text{вых}$, В},
			xlabel near ticks,
			ylabel near ticks,
			axis x line = middle,
			axis y line = left,
			xtick={-16, -12, -8, -4, 0, 4, 8, 12, 16},
			xmin = -16,
			xmax = 16,
			ymin = -15,
			ymax = 15,
			grid=major
		]
		\addplot+[blue, mark options={fill=blue}] table[x=e_c,y=u_out,col sep=comma]{data/2_k_dc_plus.csv};
		\addplot+[blue, mark options={fill=blue}] table[x=e_c,y=u_out,col sep=comma]{data/2_k_dc_minus.csv};
		\addplot +[mark=none, red, samples=2] coordinates {(-15,-15) (15,15)};
		\legend{Эксп., ,Teор.}
		\end{axis}
	\end{tikzpicture}
	\caption{Зависимость напряжения $U_\text{вых}$ от $e_c$ инвертирующего усилителя}
\end{center}
\end{figure}

Среднее значение коэффициента передачи при положительном и отрицательном входном сигнале $\bar{K} \approx 0.991$.

\subsubsection{Перменный входной сигнал}

\begin{table}[H]
\begin{center}
	\caption{Зависимость напряжения $U_\text{вых}$ от $e_c$ инвертирующего усилителя при переменном входном сигнале}
	\def\tabcolsep{40pt}
	\def\arraystretch{1.2}
	\pgfplotstabletypeset[col sep=comma,
	    columns={e_c,u_out,k},
	    column type/.add={|c|}{},
	    columns/e_c/.style={fixed, precision=3, zerofill, column name={$e_c$, В}},
	    columns/u_out/.style={fixed, precision=3, zerofill, column name={$U_\text{вых}$, В}},
	    columns/k/.style={fixed, precision=3, zerofill, column name={$K$}},
	    every nth row={1}{before row=\hline},
	    every head row/.style={before row=\hline, after row=\hline},
	    every last row/.style={after row=\hline}
	   ]{data/2_k_ac.csv}
\end{center}
\end{table}

Среднее значение $\bar{K} \approx 0.990$.

Итого, среднее значение коэффициента передачи при положитеьном, отрицательном и переменном входном сигнале $|\bar{K}| \approx 0.991$

\begin{figure}[H]
\begin{center}
	\begin{tikzpicture} [every plot/.append style={thick}]
		\begin{axis}[
			height=0.45\textheight,
			width=0.95\textwidth,
			legend pos=north west,
			xlabel={$e_c$, В},
			ylabel={$U_\text{вых}$, В},
			xlabel near ticks,
			ylabel near ticks,
			axis x line = middle,
			axis y line = left,
			xmin = 0,
			xmax = 12,
			ymin = 0,
			ymax = 12,
			grid=major
		]
		\addplot table[x=e_c,y=u_out,col sep=comma]{data/2_k_ac.csv};
		\addplot +[mark=none, red, samples=2] coordinates {(1,1) (11.5,11.5)};
		\addplot +[mark=none, dashed, black, samples=2] coordinates {(9.8,0) (9.8,12)};
		\legend{Эксп., Teор.}
		\end{axis}
	\end{tikzpicture}
	\caption{Зависимость напряжения $U_\text{вых}$ от $e_c$ инвертирующего усилителя при переменном входном сигнале}
\end{center}
\end{figure}

\subsection{Операционный усилитель без инверсии фазы с разделяющим конденсатором}

\begin{table}[H]
\begin{center}
	\caption{Зависимость $U_\text{вых}$ от $e_c$ неинвертирующего усилителя с разделяющим конденсатором при переменном входном сигнале}
	\def\tabcolsep{30pt}
	\def\arraystretch{1.2}
	\pgfplotstabletypeset[col sep=comma,
	    columns={e_c,u_out,k},
	    column type/.add={|c|}{},
	    columns/e_c/.style={fixed, precision=3, zerofill, column name={$e_c$, В}},
	    columns/u_out/.style={fixed, precision=3, zerofill, column name={$U_\text{вых}$, В}},
	    columns/k/.style={fixed, precision=3, zerofill, column name={$K$}},
	    every nth row={1}{before row=\hline},
	    every head row/.style={before row=\hline, after row=\hline},
	    every last row/.style={after row=\hline}
	   ]{data/3_k.csv}
\end{center}
\end{table}

\vspace{-1cm}

$\bar{K} \approx 0.686$

\begin{figure}[H]
\begin{center}
	\begin{tikzpicture} [every plot/.append style={thick}]
		\begin{axis}[
			height=0.4\textheight,
			width=0.95\textwidth,
			legend pos=north west,
			xlabel={$e_c$, В},
			ylabel={$U_\text{вых}$, В},
			xlabel near ticks,
			ylabel near ticks,
			axis x line = middle,
			axis y line = left,
			xmin = 3,
			xmax = 14,
			ymin = 2,
			ymax = 8,
			grid=major
		]
		\addplot+[smooth] table[x=e_c,y=u_out,col sep=comma]{data/3_k.csv};
%		\addplot table[x=e_c,y=u_out,col sep=comma]{data/ampl_theory.csv};
		\legend{Эксп., Teор.}
		\end{axis}
	\end{tikzpicture}
	\caption{Зависимость $U_\text{вых}$ от $e_c$ неинвертирующего усилителя с разделяющим конденсатором при переменном входном сигнале}
\end{center}
\end{figure}



\begin{table}[H]
\begin{center}
	\caption{ЛАЧХ неинвертирующего усилителя с разделяющим конденсатором при переменном входном сигнале}
	\def\tabcolsep{20pt}
	\def\arraystretch{1.2}
	\pgfplotstabletypeset[col sep=comma,
	    columns={f,e_c,u_out,k,lgk},
	    column type/.add={|c|}{},
	    columns/f/.style={fixed, precision=3, column name={$f$, Гц}},
	    columns/e_c/.style={fixed, precision=3, zerofill, column name={$e_c$, В}},
	    columns/u_out/.style={fixed, precision=3, zerofill, column name={$U_\text{вых}$, В}},
	    columns/k/.style={fixed, precision=3, zerofill, column name={$K$}},
	    columns/lgk/.style={fixed, precision=3, zerofill, column name={$20\cdot\lg{K}$}},
	    every nth row={1}{before row=\hline},
	    every head row/.style={before row=\hline, after row=\hline},
	    every last row/.style={after row=\hline}
	   ]{data/3_k_f.csv}
\end{center}
\end{table}

\vspace{-1cm}

\begin{figure}[H]
\begin{center}
	\begin{tikzpicture} [every plot/.append style={thick}]
		\begin{axis}[
			height=0.4\textheight,
			width=0.95\textwidth,
			legend pos=south east,
			xlabel={$f$, Гц},
			ylabel={$20\cdot \lg{K}$},
			xlabel near ticks,
			ylabel near ticks,
			axis x line = middle,
			axis y line = left,
			xmode=log,
			log basis x=2,
			xmin = 2^3,
			xmax = 2^18,
			ymin = -35,
			ymax = 5,
			grid=major
		]
		\addplot+[smooth] table[x=f,y=lgk,col sep=comma]{data/3_k_f.csv};
%		\addplot table[x=e_c,y=u_out,col sep=comma]{data/ampl_theory.csv};
		\legend{Эксп., Teор.}
		\end{axis}
	\end{tikzpicture}
	\caption{ЛАЧХ неинвертирующего усилителя с разделяющим конденсатором при переменном входном сигнале}
\end{center}
\end{figure}

По 4 точкам из диапозона $[2^8\dots2^{11}]$ построим интерполяционный полином 3-й степени. Решение уравнения -- частота перегиба:

\begin{displaymath}
	-13.61 + 0.032 \cdot f - 2.64 \cdot 10^{-5} \cdot f^2 + 6.79 \cdot 10^{-9} \cdot f^3 = -3
\end{displaymath}

\begin{displaymath}
	f = 529 \text{ Гц}
\end{displaymath}


\section{Погрешности}

\subsection{Операционный усилитель с инверсией фазы}

\begin{displaymath}
	\delta_{max} K = \sqrt{(\delta R_1)^2 + (\delta R_2)^2} =  \sqrt{0.1^2 + 0.1^2} = 0.141 = 14.1 \%
\end{displaymath}

\begin{displaymath}
	\delta K = \left|\frac{K_\text{теор.} - K_\text{эксп.}}{K_\text{теор.}} \right| = \left|\frac{1 - 0.984}{1}\right| = 0.016 = 1.6\% < \delta_{max} K = 14.1\%
\end{displaymath}

\subsection{Операционный усилитель без инверсии фазы}

\begin{displaymath}
\begin{aligned}
	\delta_{max} K = \sqrt{(\delta R_1)^2 + (\delta R_2)^2 + (\delta R_3)^2 + (\delta R_4)^2} = \sqrt{0.1^2 + 0.1^2 + 0.1^2 + 0.1^2} = \\ = 0.2 = 20 \%
\end{aligned}
\end{displaymath}

\begin{displaymath}
	\delta K = \left|\frac{K_\text{теор.} - K_\text{эксп.}}{K_\text{теор.}} \right| = \left|\frac{1 - 0.991}{1}\right| = 0.009 = 0.9\% < \delta_{max} K = 20\%
\end{displaymath}

\subsection{Операционный усилитель без инверсии фазы с разделяющим конденсатором}

\begin{displaymath}
\begin{aligned}
	\delta_{max} K = \sqrt{(\delta R_1)^2 + (\delta R_2)^2 + (\delta R_3)^2 + (\delta R_4)^2 + (\delta C_p)^2} = \\ = \sqrt{0.1^2 + 0.1^2 + 0.1^2 + 0.1^2 + 0.1^2} = 0.224 = 22.4 \%
\end{aligned}
\end{displaymath}

\begin{displaymath}
	\delta_{max} f = \sqrt{(\delta R_3)^2 + (\delta R_4)^2 + (\delta C_p)^2} =  \sqrt{0.1^2 + 0.1^2 + 0.1^2} = 0.173 = 17.3 \%
\end{displaymath}

\begin{displaymath}
	\delta K = \left|\frac{K_\text{теор.} - K_\text{эксп.}}{K_\text{теор.}} \right| = \left|\frac{1 - 0.686}{1}\right| = 0.314 = 31.4\% \approx \delta_{max} K = 22.4\%
\end{displaymath}

\begin{displaymath}
	\delta f = \left|\frac{f_\text{теор.} - f_\text{эксп.}}{f_\text{теор.}} \right| = \left|\frac{500 - 529}{500}\right| = 0.058 = 5.8\% < \delta_{max} f = 17.3\%
\end{displaymath}

\section{Выводы}

Вычесленные уточнённые теоретические и полученные экспериментальные значения близки. Следовательно, формулы \ref{eq:k1} -- \ref{eq:k2} могут быть использованы при расчетах.

\end{document}