\documentclass[a4paper,14pt]{extarticle}

\usepackage[utf8x]{inputenc}
\usepackage[T1,T2A]{fontenc}
\usepackage[russian]{babel}
\usepackage{hyperref}
\usepackage{indentfirst}
\usepackage{here}
\usepackage{array}
\usepackage{graphicx}
\usepackage{caption}
\usepackage{subcaption}
\usepackage{chngcntr}
\usepackage{amsmath}
\usepackage{amssymb}
\usepackage{pgfplots}
\usepackage{pgfplotstable}
\usepackage[left=2cm,right=2cm,top=2cm,bottom=2cm,bindingoffset=0cm]{geometry}

\renewcommand{\le}{\ensuremath{\leqslant}}
\renewcommand{\leq}{\ensuremath{\leqslant}}
\renewcommand{\ge}{\ensuremath{\geqslant}}
\renewcommand{\geq}{\ensuremath{\geqslant}}
\renewcommand{\epsilon}{\ensuremath{\varepsilon}}
\renewcommand{\phi}{\ensuremath{\varphi}}

\counterwithin{figure}{section}
\counterwithin{equation}{section}
\counterwithin{table}{section}
\newcommand{\sign}[1][5cm]{\makebox[#1]{\hrulefill}} % Поля подписи и даты
\graphicspath{{pics/}} % Путь до папки с картинками
\captionsetup{justification=centering,margin=1cm}
\def\arraystretch{1.3}

\begin{document}	% начало документа

\begin{titlepage}
\begin{center}
	\textbf{Санкт-Петербургский Политехнический Университет \\Петра Великого}\\[0.3cm]
	\small Институт компьютерных наук и технологий \\[0.3cm]
	\small Кафедра компьютерных систем и программных технологий\\[4cm]
	
	\textbf{ОТЧЕТ}\\ \textbf{о лабораторной работе}\\[0.5cm]
	\textbf{<<Исследование однокаскадных транзисторных усилителей>>}\\[0.1cm]
	\textbf{Электротехника и Электроника}\\[10.5cm]
\end{center}

\begin{flushright}
	\begin{minipage}{0.60\textwidth}
		\begin{flushleft}
			\small \textbf{Работу выполнили студенты}\\[3mm]
			\small группа 23501/4 \hspace*{17mm} Дьячков В.В.\\[3mm]
			\small группа 23501/4 \hspace*{17mm} Ламтев А.Ю.\\[5mm]
			
			\small \textbf{Преподаватель}\\[5mm]
		 	\small \sign[3.5cm] \hspace*{8mm} к.т.н., доц. Кочетков Ю.Д.\\[0.5cm]
		\end{flushleft}
	\end{minipage}
\end{flushright}

\vfill

\begin{center}
	\small Санкт-Петербург\\
	\small \the\year
\end{center}
\end{titlepage}


\section{Цель работы}

Ознакомиться с принципом действия параметрического стабилизатора напряжения. Настроить и исследовать его. Рассчитать силовые звенья.

\section{Чертеж схемы исследуемого устройства}

\begin{figure}[h]
\centering
\begin{subfigure}[b]{0.45\textwidth}
\includegraphics[scale=0.35]{img/stabilizator.png}
\caption{Параметрический стабилизатор\\ постоянного напряжения}\label{figure:2.1:a}
\end{subfigure}
\begin{subfigure}[b]{0.45\textwidth}
\includegraphics[scale=0.35]{img/substitution.png}
\caption{Схема замещения \\стабилизатора}\label{figure:2.1:b}
\end{subfigure}
\caption{}\label{figure:2.1}
\end{figure}


\section{Исходные данные}

\begin{table}[H]
	\begin{center}
	\caption{Исходные данные}
	\def\arraystretch{1.2}
		\begin{tabular}{|c|c|}
		\hline 
		$U_\text{вх}$ & 15 В \\ 
		\hline 
		$U_\text{вых}$ & 9 В \\ 
		\hline 
		$I_p$ & 8 мА \\ 
		\hline 
		$R_\text{н}$ & 500 Ом \\ 
		\hline 
		\end{tabular} 
		\label{tab:3:1}
	\end{center}
\end{table}

\begin{table}[H]
	\begin{center}
	\caption{Характеристика стабилитрона кс156А}
	\def\arraystretch{1.2}
		\begin{tabular}{|c|c|}
		\hline 
		$U$ & 5.6 В \\ 
		\hline 
		$I_{\text{ст}\ min}$ & 3 мА \\ 
		\hline 
		$I_{\text{ст}\ max}$ & 50 мА \\ 
		\hline 
		\end{tabular} 
		\label{tab:3:1}
	\end{center}
\end{table}

\section{Расчет параметров элементов и характеристик стабилизатора}
%\begin{flalign*}
$I_\text{вых} = \frac{U_\text{вых}}{R_\text{н}} = \frac{9}{500} = 0.018$ A
%\end{flalign*}
$R_\text{бал} = \frac{U_\text{вх} - U_\text{вых}}{I_p + I_\text{вых}} = \frac{15 - 9}{0.008 + 0.018} = 230.769$ Ом

$U_{\text{вх}\ min} = U_\text{вых} + R_\text{бал} \cdot (I_{\text{ст}\ min} + I_\text{вых}) = 9 + 230.769 \cdot (0.003 + 0.018) = \\[1mm] = 13.846$ В

$U_{\text{вх}\ max} = U_\text{вых} + R_\text{бал} \cdot (I_{\text{ст}\ max} + I_\text{вых}) = 9 + 230.769 \cdot (0.05 + 0.018) = \\[1mm] = 24.692$ В

$I_{\text{вых}\ min} = 0$

$I_{\text{вых}\ max} = \frac{U_\text{вх} - U_\text{вых} - I_{\text{ст}\ min} \cdot R_\text{бал}}{R_\text{бал}} = \frac{15 - 9 - 0.003 \cdot 230.769}{230.769} = 0.023$ А

$R_{\text{н}\ min} = \frac{U_\text{вых}}{I_{\text{вых}\ max}} = \frac{9}{0.023} = 391$ Ом

$R_{\text{н}\ max} = \frac{U_\text{вых}}{I_{\text{вых}\ min}} = \frac{9}{0} = \infty $ Ом 


\section{Теоретические зависимости}


\section{Экспериментально снятые зависимости}



\section{Погрешности}

\subsection{Предельно допустимые погрешности}

\begin{center}
$\delta R = 0.1 = 10\%$\\
$\delta C = 0.1 = 10\%$\\
\end{center}

Дифференцирующая и интегрирующая цепи включают в себя по одному резистору и по одному конденсатору, поэтому предельно допустимая погрешность $U_0$ вычисляется следующим образом:


$\delta U_0 = \sqrt{(\delta R)^2 + (\delta C)^2} = \sqrt{0.1^2 + 0.1^2} = \sqrt{0.02} = 0.141 = 14.1 \%$

\subsection{Погрешности результатов эксперемента} %заменить epsilon на delta

  
\section{Выводы}

Во всех случаях, кроме одного, приведенные погрешности вычисленных значений $U_\text{вых}$ не превышают предельно допустимую погрешность. А в этом отдельном случае превышение объясняется методической ошибкой измерения.

Таким образом, формулы для вычисления теоретических значений $U_\text{вых}$ 4.2 и 4.3 являются верными.

\end{document}
